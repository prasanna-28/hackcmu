\documentclass[12pt,a4paper]{article}
\usepackage{amsmath}
\usepackage{amssymb}
\usepackage{geometry}

\geometry{
    top=2cm,
    bottom=2cm,
    left=2cm,
    right=2cm
}

\begin{document}

\section{Inverse Matrices}

An \textit{inverse} to a matrix $A$ is another matrix $A^{-1}$ such that $AA^{-1} = A^{-1}A = I$.

\subsection{Examples}

\begin{itemize}
    \item $A = \begin{bmatrix} 2 & 1 \\ 1 & 3 \end{bmatrix}$, $B = \begin{bmatrix} 3 & -1 \\ -1 & 2 \end{bmatrix}$ are inverses
    \item Since $AB = \begin{bmatrix} 5 & 0 \\ 0 & 5 \end{bmatrix} = 5I$, we can write $B = \frac{1}{5}A^{-1}$
    \item Not all matrices have inverses. $A = \begin{bmatrix} 1 & 2 \\ 2 & 4 \end{bmatrix}$ has no inverse.
    \item Since $AB = \begin{bmatrix} 5 & 10 \\ 10 & 20 \end{bmatrix} = 5B$
\end{itemize}

\subsection{Connection to Linear Transformations}

Let $T$ be a linear transformation with associated matrix $A$ (in "standard" basis). Then the inverse transformation exists if and only if $A$ has an inverse. The matrix of the inverse transformation is $A^{-1}$.

\section{2x2 Inverse Formula}

Let $A = \begin{bmatrix} a & b \\ c & d \end{bmatrix}$, then if $ad-bc \neq 0$:

\begin{equation}
    A^{-1} = \frac{1}{ad-bc} \begin{bmatrix} d & -b \\ -c & a \end{bmatrix}
\end{equation}

This is a special case of a more general formula (involving the determinant and adjugate matrix, which are complicated to compute but follow this idea).

\subsection{Using Cross-Section to Find the Inverse}

Suppose $A$ has inverse $A^{-1}$
Then $AA^{-1} = \begin{bmatrix} 1 & 0 \\ 0 & 1 \end{bmatrix} = I$

So $A \begin{bmatrix} x_1 \\ x_2 \end{bmatrix} = \begin{bmatrix} 1 \\ 0 \end{bmatrix}$ and $A \begin{bmatrix} y_1 \\ y_2 \end{bmatrix} = \begin{bmatrix} 0 \\ 1 \end{bmatrix}$

Instead of solving $[Ax_1], [Ax_2], \ldots, [Ax_n]$ individually, we can combine these into one system and solve once:

\begin{equation}
    [A|e_1, e_2, \ldots, e_n] \leadsto [I|A^{-1}]
\end{equation}

If $A$ doesn't have $n$ pivots (one in each row), then $A$ doesn't exist.

If $A$ has $n$ pivots, then the algorithm will produce $A^{-1}$.

\subsection{Summary}

\begin{itemize}
    \item If $A$ doesn't have $n$ pivots, then $A^{-1}$ doesn't exist
    \item If $A$ has $n$ pivots, then the RREF is $[I|A^{-1}]$
\end{itemize}

\section{Example}

Let $A = \begin{bmatrix} 2 & 1 & 0 \\ 1 & 2 & 1 \\ 0 & 1 & 2 \end{bmatrix}$. Find $A^{-1}$, if it exists.

\begin{align*}
    \begin{bmatrix}
    2 & 1 & 0 & | & 1 & 0 & 0 \\
    1 & 2 & 1 & | & 0 & 1 & 0 \\
    0 & 1 & 2 & | & 0 & 0 & 1
    \end{bmatrix}
    &\leadsto
    \begin{bmatrix}
    1 & 0 & 0 & | & \frac{3}{4} & -\frac{1}{4} & -\frac{1}{4} \\
    0 & 1 & 0 & | & -\frac{1}{2} & \frac{1}{2} & 0 \\
    0 & 0 & 1 & | & \frac{1}{4} & -\frac{1}{4} & \frac{1}{2}
    \end{bmatrix}
\end{align*}

Therefore, $A^{-1} = \begin{bmatrix} \frac{3}{4} & -\frac{1}{4} & -\frac{1}{4} \\ -\frac{1}{2} & \frac{1}{2} & 0 \\ \frac{1}{4} & -\frac{1}{4} & \frac{1}{2} \end{bmatrix}$ (Check: Verify $AA^{-1} = I$)

\section{Using Inverses to Solve Systems}

If $A$ exists, we can solve $Ax = b$ by multiplying by $A^{-1}$:

\begin{align*}
    A^{-1}Ax &= A^{-1}b \\
    Ix &= A^{-1}b \\
    x &= A^{-1}b
\end{align*}

However, this is not the solution $A^{-1}b$. Moreover, even if $A$ has an inverse (i.e., is invertible), this solution is unique.

\end{document}