\documentclass[12pt,a4paper]{article}
\usepackage{amsmath}
\usepackage{amssymb}
\usepackage[margin=1in]{geometry}

\begin{document}

\section{Matrix Inverses}

An \textit{inverse} of a matrix $A$ is a matrix that $AA^{-1} = A^{-1}A = I$.

\subsection{Examples}

\[ A = \begin{bmatrix} 3 & 1 \\ 2 & 2 \end{bmatrix}, B = \begin{bmatrix} \frac{1}{2} & -\frac{1}{4} \\ -\frac{1}{4} & \frac{3}{8} \end{bmatrix} \text{ are inverses} \]

Since $AB = BA = I$, we can write $B = A^{-1}$.

\subsection{Properties}

Not all matrices have inverses. For example:

\[ A = \begin{bmatrix} 0 & 0 \\ 0 & 1 \end{bmatrix} \text{ has no inverse} \]

While $AB = \begin{bmatrix} 2 & 2 \\ 2 & 2 \end{bmatrix} = B$ does.

\section{Connection to Linear Transformations}

If $T$ is a linear transformation with associated matrix $A$ (as $T(x) = Ax$), then the inverse transformation has associated matrix $A^{-1}$.

\section{2x2 Inverse Formula}

Let $A = \begin{bmatrix} a & b \\ c & d \end{bmatrix}$, then if $ad-bc \neq 0$:

\[ A^{-1} = \frac{1}{ad-bc} \begin{bmatrix} d & -b \\ -c & a \end{bmatrix} \]

This is a special case of a more general formula (involving the determinant and adjugate) that works for any size matrix. For 2x2, it's easy to verify that $AA^{-1} = I$.

\section{Using Linear Systems to Find the Inverse}

Suppose $A$ has inverse $A^{-1}$. Then $AA^{-1} = I$.

So $A \begin{bmatrix} x_1 & x_2 \\ x_3 & x_4 \end{bmatrix} = \begin{bmatrix} 1 & 0 \\ 0 & 1 \end{bmatrix}$

To find $A^{-1}$, we must solve the system $Ax = e_i$ for each column.

Instead of solving $[Ax_1] [Ax_2]$ individually, we can combine these into one augmented matrix:

\[ [A|e_1, e_2, \ldots] = [A|I] \]

If $A$ doesn't have $n$ pivots (one in each row), then $A^{-1}$ does not exist.

If $A$ has $n$ pivots, then the algorithm produces $[I|A^{-1}]$.

\section{Summary of Inverse Existence}

\begin{itemize}
    \item If $A$ doesn't have $n$ pivots, then $A^{-1}$ doesn't exist
    \item If $A$ has $n$ pivots, then $A$ is invertible and the RREF is $[I|A^{-1}]$
\end{itemize}

\section{Example: Finding $A^{-1}$}

Find $A^{-1}$ if it exists:

\[ A = \begin{bmatrix} 2 & 1 & 3 \\ 1 & 0 & 1 \\ 3 & 1 & 4 \end{bmatrix} \]

We form the augmented matrix $[A|I]$:

\[ \begin{bmatrix} 
2 & 1 & 3 & | & 1 & 0 & 0 \\
1 & 0 & 1 & | & 0 & 1 & 0 \\
3 & 1 & 4 & | & 0 & 0 & 1
\end{bmatrix} \]

After row operations to get RREF:

\[ \begin{bmatrix} 
1 & 0 & 0 & | & -1 & 3 & -1 \\
0 & 1 & 0 & | & 3 & -8 & 3 \\
0 & 0 & 1 & | & 1 & -2 & 1
\end{bmatrix} \]

Therefore:

\[ A^{-1} = \begin{bmatrix} -1 & 3 & -1 \\ 3 & -8 & 3 \\ 1 & -2 & 1 \end{bmatrix} \]

\section{Using Inverses to Solve Systems}

If $A$ exists, we can solve $Ax = b$ by multiplying both sides by $A^{-1}$:

\begin{align*}
A^{-1}Ax &= A^{-1}b \\
Ix &= A^{-1}b \\
x &= A^{-1}b
\end{align*}

Note: This is not the solution $A \rightarrow R$. However, given $A$ has $n$ pivots (is invertible), this solution is unique.

\end{document}