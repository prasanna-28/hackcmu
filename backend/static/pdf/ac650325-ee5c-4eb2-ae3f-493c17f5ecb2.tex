\documentclass[12pt,a4paper]{article}
\usepackage[margin=1in]{geometry}
\usepackage{amsmath}
\usepackage{amssymb}

\begin{document}

\section{Set Theory}

\subsection{Basic Definitions}

\begin{align*}
\mathbb{N} &= \{0, 1, 2, \ldots\} & \text{natural numbers} \\
\mathbb{Z} &= \{\ldots, -1, 0, 1, \ldots\} & \text{integers} \\
\mathbb{Q} &= \{\frac{a}{b} \mid a,b \in \mathbb{Z}, b \neq 0\} & \text{rationals} \\
\mathbb{R} &= \text{real numbers} \\
\mathbb{C} &= \{a + bi \mid a,b \in \mathbb{R}\} & \text{complex numbers}
\end{align*}

\textit{Interval notation}: $(a,b) = \{x \in \mathbb{R} \mid a < x < b\}$

\textit{Set-builder notation}: $\{x \mid P(x)\}$ where $P(x)$ is a property of $x$

\subsection{Set Operations}

\begin{align*}
A \cup B &= \{x \mid x \in A \text{ or } x \in B\} & \text{union} \\
A \cap B &= \{x \mid x \in A \text{ and } x \in B\} & \text{intersection}
\end{align*}

\textit{Venn diagrams} are used to visualize set operations.

\begin{align*}
A \setminus B &= \{x \in A \mid x \notin B\} & \text{set difference} \\
A \triangle B &= (A \cup B) \setminus (A \cap B) & \text{symmetric difference}
\end{align*}

\subsection{Properties}

Let $A_\alpha$ be an indexed family of sets (where $\alpha \in I$).

\begin{align*}
\bigcup_{\alpha \in I} A_\alpha &= \{x \mid x \in A_\alpha \text{ for some } \alpha \in I\} \\
\bigcap_{\alpha \in I} A_\alpha &= \{x \mid x \in A_\alpha \text{ for every } \alpha \in I\}
\end{align*}

\textit{De Morgan's Laws}:
\begin{align*}
\overline{\bigcup_{\alpha \in I} A_\alpha} &= \bigcap_{\alpha \in I} \overline{A_\alpha} \\
\overline{\bigcap_{\alpha \in I} A_\alpha} &= \bigcup_{\alpha \in I} \overline{A_\alpha}
\end{align*}

\subsection{Functions}

A \textit{function} is a mapping or transformation from one set to another.

Let $A, B$ be sets. $A \to B$ denotes a function from $A$ to $B$.

\textit{Injective} (one-to-one): If $f(x_1) = f(x_2)$, then $x_1 = x_2$.

\textit{Surjective} (onto): For every $y \in B$, there exists $x \in A$ such that $f(x) = y$.

\textit{Bijective}: Both injective and surjective.

If $f: A \to B$ is bijective, then there exists an inverse function $f^{-1}: B \to A$.

\subsection{Composition of Functions}

If $f: A \to B$ and $g: B \to C$, then the composition $g \circ f: A \to C$ is defined by $(g \circ f)(x) = g(f(x))$.

\end{document}