\documentclass[12pt,a4paper]{article}
\usepackage{amsmath}
\usepackage[margin=1in]{geometry}

\begin{document}

\section{Inverse Matrices}

An \textit{inverse matrix} $A$ is invertible if there is a matrix $A^{-1}$ such that $AA^{-1} = A^{-1}A = I$.

\subsection{Examples}

$A = \begin{bmatrix} 3 & 1 \\ 2 & 1 \end{bmatrix}$, $B = \begin{bmatrix} 1 & -1 \\ -2 & 3 \end{bmatrix}$ are inverses

Since $AB = \begin{bmatrix} 1 & 0 \\ 0 & 1 \end{bmatrix}$, we can write $B = A^{-1}$

Not all matrices have inverses:

$A = \begin{bmatrix} 0 & 0 \\ 1 & 1 \end{bmatrix}$ has no inverse

Since $AB = \begin{bmatrix} 0 & 0 \\ a+b & a+b \end{bmatrix} \neq I$

\subsection{Connection to Linear Transformations}

$A^{-1}$ is the \textit{inverse transformation} with respect to $A$ (or $T_A^{-1}$ w.r.t. $T_A$).

The inverse of $T$ is a linear transformation whose associated matrix is $A^{-1}$.

\section{2x2 Inverse Formula}

Let $A = \begin{bmatrix} a & b \\ c & d \end{bmatrix}$, then if $ad-bc \neq 0$:

\[A^{-1} = \frac{1}{ad-bc} \begin{bmatrix} d & -b \\ -c & a \end{bmatrix}\]

This is a special case of a more general formula (which is complicated to compute but has the same idea for matrices of any size).

\subsection{Using Gauss-Jordan to Find the Inverse}

Suppose $A$ has inverse $A^{-1}$.
Then $AA^{-1} = I$
So $A[A^{-1}] = [I]$

To find $A^{-1}$, we need to solve the system $Ax = e_i$ for each $i$.

Instead of solving $[Ax][Ax][Ax]$ individually, we can combine these systems into one matrix equation:

\[[A|e_1, e_2, ..., e_n] \to [I|A^{-1}]\]

If $A$ doesn't have $n$ pivots (one in each row), then $A^{-1}$ does not exist.

If $A$ has $n$ pivots, then the algorithm produces $A^{-1}$.

\subsection{Summary}
\begin{itemize}
    \item If $A$ doesn't have $n$ pivots, then $A^{-1}$ doesn't exist
    \item If $A$ has $n$ pivots, then $A$ is invertible and the RREF is $[I|A^{-1}]$
\end{itemize}

\section{Example}

Find $A^{-1}$, if it exists:

\[A = \begin{bmatrix} 2 & 1 & 3 \\ 1 & 0 & 1 \\ 3 & 1 & 4 \end{bmatrix}\]

\[
\begin{bmatrix}
2 & 1 & 3 & | & 1 & 0 & 0 \\
1 & 0 & 1 & | & 0 & 1 & 0 \\
3 & 1 & 4 & | & 0 & 0 & 1
\end{bmatrix}
\]

After row operations:

\[
\begin{bmatrix}
1 & 0 & 0 & | & -1 & 3 & -1 \\
0 & 1 & 0 & | & 3 & -8 & 3 \\
0 & 0 & 1 & | & 1 & -2 & 1
\end{bmatrix}
\]

Therefore:

\[A^{-1} = \begin{bmatrix} -1 & 3 & -1 \\ 3 & -8 & 3 \\ 1 & -2 & 1 \end{bmatrix} \quad \text{(Check: $AA^{-1} = I$)}\]

\section{Using Inverses to Solve Systems}

If $A$ exists, we can solve $Ax = b$ by multiplying both sides by $A^{-1}$:

\begin{align*}
A^{-1}Ax &= A^{-1}b \\
Ix &= A^{-1}b \\
x &= A^{-1}b
\end{align*}

However, this solution is $O(n^3)$ (as matrix multiplication is $O(n^3)$), which is inefficient.

\end{document}