\documentclass[12pt,a4paper]{article}
\usepackage{amsmath}
\usepackage{amssymb}
\usepackage{geometry}

\geometry{
    top=2cm,
    bottom=2cm,
    left=2cm,
    right=2cm
}

\begin{document}

\section{Matrix Inverses}

An \textit{inverse} of a matrix $A$ is a matrix that $AA^{-1} = A^{-1}A = I$.

\subsection{Examples}

\begin{itemize}
    \item $A = \begin{bmatrix} 2 & 1 \\ 1 & 3 \end{bmatrix}$, $B = \begin{bmatrix} 3 & -1 \\ -1 & 2 \end{bmatrix}$ are inverses
    \item Since $AB = \begin{bmatrix} 1 & 0 \\ 0 & 1 \end{bmatrix}$, we can write $B = A^{-1}$
    \item Not all matrices have inverses: $A = \begin{bmatrix} 1 & 2 \\ 2 & 4 \end{bmatrix}$ has no inverse
\end{itemize}

\subsection{Connection to Linear Transformations}

If $T$ is a linear transformation with standard matrix $A$ (so $T(x) = A\vec{x}$), then the inverse transformation exists if and only if $A^{-1}$ exists.

\section{2x2 Inverse Formula}

Let $A = \begin{bmatrix} a & b \\ c & d \end{bmatrix}$. Then if $ad-bc \neq 0$:

\[A^{-1} = \frac{1}{ad-bc} \begin{bmatrix} d & -b \\ -c & a \end{bmatrix}\]

This is a special case of a more general formula (involving the determinant and adjugate) that works for any size matrix.

\subsection{Using Gauss-Jordan to Find the Inverse}

Suppose $A$ has inverse $A^{-1}$. Then:
\[AA^{-1} = I\]
\[A[A^{-1}] = [I]\]

So to find $A^{-1}$, we need to solve the system $Ax = e_i$ for each standard basis vector $e_i$.

Instead of solving $[A|e_1], [A|e_2], \ldots, [A|e_n]$ individually, we can streamline the process by solving $[A|I]$ in one go.

\subsection{Invertibility Conditions}

If $A$ is $n \times n$, then the following are equivalent:
\begin{itemize}
    \item $A$ has an inverse
    \item $Ax = 0$ has only the trivial solution
    \item The RREF of $A$ is $I$
    \item $A$ has $n$ pivots
    \item $\det A \neq 0$
\end{itemize}

\section{Example: Finding $A^{-1}$}

Let $A = \begin{bmatrix} 2 & 1 & 0 \\ 1 & 2 & 1 \\ 0 & 1 & 2 \end{bmatrix}$. Find $A^{-1}$, if it exists.

\begin{align*}
    [A|I] &= \begin{bmatrix}
    2 & 1 & 0 & | & 1 & 0 & 0 \\
    1 & 2 & 1 & | & 0 & 1 & 0 \\
    0 & 1 & 2 & | & 0 & 0 & 1
    \end{bmatrix} \\
    &\sim \begin{bmatrix}
    1 & 0 & 0 & | & \frac{3}{4} & -\frac{1}{4} & -\frac{1}{8} \\
    0 & 1 & 0 & | & -\frac{1}{2} & \frac{1}{2} & -\frac{1}{4} \\
    0 & 0 & 1 & | & \frac{1}{8} & -\frac{1}{4} & \frac{5}{8}
    \end{bmatrix}
\end{align*}

Therefore, 
\[A^{-1} = \begin{bmatrix}
\frac{3}{4} & -\frac{1}{4} & -\frac{1}{8} \\
-\frac{1}{2} & \frac{1}{2} & -\frac{1}{4} \\
\frac{1}{8} & -\frac{1}{4} & \frac{5}{8}
\end{bmatrix}\]

\subsection{Using Inverses to Solve Systems}

If $A$ is invertible, we can solve $Ax = b$ by multiplying both sides by $A^{-1}$:

\begin{align*}
    A^{-1}Ax &= A^{-1}b \\
    Ix &= A^{-1}b \\
    x &= A^{-1}b
\end{align*}

However, this is not an efficient method for solving systems in practice.

\end{document}