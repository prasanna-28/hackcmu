\documentclass[12pt,a4paper]{article}
\usepackage[margin=1in]{geometry}
\usepackage{amsmath}

\begin{document}

\section{Matrix Inverses}

\subsection{Definition}

An $n \times n$ matrix $A$ is invertible if there is a matrix $A^{-1}$ such that $AA^{-1} = A^{-1}A = I$.

\subsection{Properties}

\begin{itemize}
    \item $A = \begin{bmatrix} 2 & 1 \\ 1 & 3 \end{bmatrix}$, $B = \begin{bmatrix} 3 & 2 \\ 1 & 4 \end{bmatrix}$ are inverses
    \item Since $AB = \begin{bmatrix} 1 & 0 \\ 0 & 1 \end{bmatrix}$, we can write $B = A^{-1}$
    \item Not all $n \times n$ matrices have inverses
    \item $A = \begin{bmatrix} 0 & 0 \\ 0 & 1 \end{bmatrix}$ has no inverse
    \item Since $AB = \begin{bmatrix} 0 & 0 \\ 0 & 1 \end{bmatrix} \neq I$
\end{itemize}

\subsection{Connection to Linear Transformations}

$A^{-1}$ is the \textit{inverse transformation} with respect to $A$ (or "undo $A$").

If the inverse of $A$ exists, then the inverse transformation also exists.

\section{2x2 Inverse Formula}

Let $A = \begin{bmatrix} a & b \\ c & d \end{bmatrix}$, then if $ad-bc \neq 0$:

\[A^{-1} = \frac{1}{ad-bc} \begin{bmatrix} d & -b \\ -c & a \end{bmatrix}\]

\textit{Note}: $ad-bc$ is called the \textit{determinant} of $A$, often written as $\det(A)$.

There is a general $n \times n$ inverse formula (involving the determinant and \textit{adjugate matrix}), but it's complicated to compute (so it's slow).

\subsection{Using Gauss-Jordan to Find the Inverse}

Suppose $A$ has inverse $A^{-1}$.
Then $AA^{-1} = I$.
So $A[A^{-1}] = [I]$.

To find $A^{-1}$, we need to solve the system $Ax = e_i$ for each $i$.

Instead of solving $[Ax_1] [Ax_2] \ldots [Ax_n]$ individually, we can combine these into one matrix equation:

\[[A|e_1, e_2, \ldots] = [A|I]\]

If $A$ doesn't have $n$ pivots (one in each row), then $A$ doesn't exist.

If $A$ has $n$ pivots, then the algorithm produces $[I|A^{-1}]$.

\subsection{Summary}

\begin{itemize}
    \item If $A$ doesn't have $n$ pivots, then $A^{-1}$ doesn't exist
    \item If $A$ has $n$ pivots, then $A^{-1}$ exists and the RREF is $[I|A^{-1}]$
\end{itemize}

\section{Example}

Find $A^{-1}$, if it exists:

\[A = \begin{bmatrix} 2 & 1 & 0 \\ 0 & 1 & 2 \\ 1 & 0 & 1 \end{bmatrix}\]

\begin{align*}
    \begin{bmatrix} 
    2 & 1 & 0 & | & 1 & 0 & 0 \\
    0 & 1 & 2 & | & 0 & 1 & 0 \\
    1 & 0 & 1 & | & 0 & 0 & 1
    \end{bmatrix}
    &\xrightarrow{R_3 - \frac{1}{2}R_1}
    \begin{bmatrix} 
    2 & 1 & 0 & | & 1 & 0 & 0 \\
    0 & 1 & 2 & | & 0 & 1 & 0 \\
    0 & -\frac{1}{2} & 1 & | & -\frac{1}{2} & 0 & 1
    \end{bmatrix} \\
    &\xrightarrow{R_3 + \frac{1}{2}R_2}
    \begin{bmatrix} 
    2 & 1 & 0 & | & 1 & 0 & 0 \\
    0 & 1 & 2 & | & 0 & 1 & 0 \\
    0 & 0 & 2 & | & -\frac{1}{2} & \frac{1}{2} & 1
    \end{bmatrix} \\
    &\xrightarrow{R_3 \cdot \frac{1}{2}}
    \begin{bmatrix} 
    2 & 1 & 0 & | & 1 & 0 & 0 \\
    0 & 1 & 2 & | & 0 & 1 & 0 \\
    0 & 0 & 1 & | & -\frac{1}{4} & \frac{1}{4} & \frac{1}{2}
    \end{bmatrix} \\
    &\xrightarrow{R_2 - 2R_3}
    \begin{bmatrix} 
    2 & 1 & 0 & | & 1 & 0 & 0 \\
    0 & 1 & 0 & | & \frac{1}{2} & \frac{1}{2} & -1 \\
    0 & 0 & 1 & | & -\frac{1}{4} & \frac{1}{4} & \frac{1}{2}
    \end{bmatrix} \\
    &\xrightarrow{R_1 - R_2}
    \begin{bmatrix} 
    2 & 0 & 0 & | & \frac{1}{2} & -\frac{1}{2} & 1 \\
    0 & 1 & 0 & | & \frac{1}{2} & \frac{1}{2} & -1 \\
    0 & 0 & 1 & | & -\frac{1}{4} & \frac{1}{4} & \frac{1}{2}
    \end{bmatrix} \\
    &\xrightarrow{R_1 \cdot \frac{1}{2}}
    \begin{bmatrix} 
    1 & 0 & 0 & | & \frac{1}{4} & -\frac{1}{4} & \frac{1}{2} \\
    0 & 1 & 0 & | & \frac{1}{2} & \frac{1}{2} & -1 \\
    0 & 0 & 1 & | & -\frac{1}{4} & \frac{1}{4} & \frac{1}{2}
    \end{bmatrix}
\end{align*}

Therefore:

\[A^{-1} = \begin{bmatrix} 
\frac{1}{4} & -\frac{1}{4} & \frac{1}{2} \\
\frac{