\documentclass[12pt,a4paper]{article}
\usepackage[margin=1in]{geometry}
\usepackage{amsmath}
\usepackage{amssymb}

\begin{document}

\section{Matrix Inverses}

An invertible matrix $A$ is one that has an \textit{inverse} $A^{-1}$ such that $AA^{-1} = A^{-1}A = I$.

\subsection{Examples}

\begin{align*}
A = \begin{bmatrix} 2 & 3 \\ 1 & 2 \end{bmatrix}, \quad B = \begin{bmatrix} 2 & 3 \\ 2 & 3 \end{bmatrix}
\end{align*}

$A$ is invertible, but $B$ is not as we can write $B = A \begin{bmatrix} 1 \\ 1 \end{bmatrix}$.

Not all matrices have inverses. For example:
\begin{align*}
C = \begin{bmatrix} 0 & 0 \\ 0 & 1 \end{bmatrix} \text{ has no inverse}
\end{align*}

\subsection{Connection to Linear Transformations}

$A^{-1}$ is the inverse transformation with respect to $A$ (as $T$ is to $V$). The inverse of $T$ is a linear transformation that "undoes" the original transformation.

\section{2x2 Inverse Formula}

Let $A = \begin{bmatrix} a & b \\ c & d \end{bmatrix}$. Then if $ad-bc \neq 0$:

\begin{equation}
A^{-1} = \frac{1}{ad-bc} \begin{bmatrix} d & -b \\ -c & a \end{bmatrix}
\end{equation}

This is a special case of Cramer's formula (which is complicated to compute but has the same form for any size matrix).

\subsection{Using Cross-Section to Find the Inverse}

Suppose $A$ has inverse $A^{-1}$:
\begin{align*}
AA^{-1} &= \begin{bmatrix} a & b \\ c & d \end{bmatrix} \begin{bmatrix} x & y \\ z & w \end{bmatrix} = I \\
A^{-1} &= \begin{bmatrix} x & y \\ z & w \end{bmatrix}
\end{align*}

So to find $A^{-1}$, we must solve the system $Av = e_i$ where $e_i$ are the standard basis vectors.

Instead of solving $[Ax] [Ay] = [e_1] [e_2]$ individually, we can combine them into one matrix equation:

\begin{equation}
[A|e_1, e_2, \ldots] = [A|I]
\end{equation}

If $A$ doesn't have an inverse (i.e., is singular), this process will fail at some point.

If $A$ has an inverse, then the algorithm:
\begin{equation}
[A|I] \rightarrow [I|A^{-1}]
\end{equation}

\section{Example: Finding $A^{-1}$}

Let $A = \begin{bmatrix} 2 & 1 & 3 \\ 1 & 0 & 1 \\ 0 & 1 & 2 \end{bmatrix}$. Find $A^{-1}$, if it exists.

\begin{align*}
\left[\begin{array}{ccc|ccc}
2 & 1 & 3 & 1 & 0 & 0 \\
1 & 0 & 1 & 0 & 1 & 0 \\
0 & 1 & 2 & 0 & 0 & 1
\end{array}\right]
&\xrightarrow{R_2 - \frac{1}{2}R_1}
\left[\begin{array}{ccc|ccc}
2 & 1 & 3 & 1 & 0 & 0 \\
0 & -\frac{1}{2} & -\frac{1}{2} & -\frac{1}{2} & 1 & 0 \\
0 & 1 & 2 & 0 & 0 & 1
\end{array}\right] \\
&\xrightarrow{R_3 + \frac{1}{2}R_2}
\left[\begin{array}{ccc|ccc}
2 & 1 & 3 & 1 & 0 & 0 \\
0 & -\frac{1}{2} & -\frac{1}{2} & -\frac{1}{2} & 1 & 0 \\
0 & \frac{3}{4} & \frac{7}{4} & \frac{1}{4} & \frac{1}{2} & 1
\end{array}\right] \\
&\xrightarrow{\text{various steps}}
\left[\begin{array}{ccc|ccc}
1 & 0 & 0 & \frac{3}{4} & -\frac{1}{2} & -\frac{1}{4} \\
0 & 1 & 0 & -\frac{1}{2} & -1 & \frac{1}{2} \\
0 & 0 & 1 & \frac{1}{4} & \frac{1}{2} & -\frac{1}{4}
\end{array}\right]
\end{align*}

Therefore:

\begin{equation}
A^{-1} = \begin{bmatrix}
\frac{3}{4} & -\frac{1}{2} & -\frac{1}{4} \\
-\frac{1}{2} & -1 & \frac{1}{2} \\
\frac{1}{4} & \frac{1}{2} & -\frac{1}{4}
\end{bmatrix}
\end{equation}

\subsection{Using Inverses to Solve Systems}

If $A$ exists, we can solve $Ax = b$ by multiplying both sides by $A^{-1}$:

\begin{align*}
A^{-1}Ax &= A^{-1}b \\
x &= A^{-1}b
\end{align*}

However, if $A$ has $n$ pivots (i.e., is invertible), this solution is unique.

\end{document}