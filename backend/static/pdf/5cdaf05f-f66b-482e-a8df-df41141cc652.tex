\documentclass[12pt,a4paper]{article}
\usepackage{amsmath}
\usepackage{amssymb}
\usepackage[margin=1in]{geometry}

\begin{document}

\section{Inverse Matrices}

An inverse matrix A is invertible if there is a matrix B such that AB = BA = I.

\subsection{Examples}

\begin{center}
$A = \begin{bmatrix} 3 & 1 \\ 2 & 1 \end{bmatrix}$, $B = \begin{bmatrix} 1 & -1 \\ -2 & 3 \end{bmatrix}$ are inverses
\end{center}

Since AB = BA = I, we can write B = $A^{-1}$

Not all matrices have inverses:

\begin{center}
$A = \begin{bmatrix} 0 & 0 \\ 3 & 0 \end{bmatrix}$ has no inverse
\end{center}

\subsection{Connection to Linear Transformations}

Let T be a linear transformation with standard matrix A (in $\mathbb{R}^n \to \mathbb{R}^n$). 
The inverse of T is a linear transformation with standard matrix $A^{-1}$.

\section{2x2 Inverse Formula}

Let $A = \begin{bmatrix} a & b \\ c & d \end{bmatrix}$, then if $ad-bc \neq 0$:

\begin{center}
$A^{-1} = \frac{1}{ad-bc} \begin{bmatrix} d & -b \\ -c & a \end{bmatrix}$
\end{center}

This is a special case of a more general formula (involving the determinant and adjugate matrix) which is more complicated to compute (but we'll see later).

\subsection{Using Cross-Section to Find the Inverse}

Suppose A has inverse A'. Then AA' = I.

So $A \begin{bmatrix} x_1 & x_2 \\ y_1 & y_2 \end{bmatrix} = \begin{bmatrix} 1 & 0 \\ 0 & 1 \end{bmatrix}$

To find A', we must solve the system Ax = e1 and Ay = e2, 
where e1, e2 are the standard basis vectors.

Instead of solving [Ax], [Ay], [...] individually, we can combine them into one augmented matrix:

\begin{center}
$[A|e_1, e_2, ...]$ = $[A|I]$
\end{center}

If A is invertible, this reduces to $[I|A^{-1}]$

\section{Summary: Finding the Inverse}

\begin{itemize}
    \item If A doesn't have n pivots, then A doesn't exist
    \item If A has n pivots, then the algorithm works and we get $A^{-1}$
\end{itemize}

\section{Example}

Find $A^{-1}$, if it exists:

\begin{center}
$A = \begin{bmatrix} 2 & 1 & 3 \\ 1 & 0 & 1 \\ 3 & 1 & 4 \end{bmatrix}$
\end{center}

\begin{align*}
[A|I] &= \begin{bmatrix} 
2 & 1 & 3 & | & 1 & 0 & 0 \\
1 & 0 & 1 & | & 0 & 1 & 0 \\
3 & 1 & 4 & | & 0 & 0 & 1
\end{bmatrix} \\[10pt]
&\sim \begin{bmatrix}
1 & 0 & 1 & | & 0 & 1 & 0 \\
0 & 1 & 1 & | & 1 & -2 & 0 \\
0 & 0 & 1 & | & -1 & 1 & 1
\end{bmatrix}
\end{align*}

Therefore:

\begin{center}
$A^{-1} = \begin{bmatrix} 1 & -1 & 1 \\ -1 & 2 & -1 \\ 1 & -1 & 0 \end{bmatrix}$ (Check: $AA^{-1} = I$)
\end{center}

\section{Using Inverses to Solve Systems}

If A exists, we can solve Ax = b by multiplying by $A^{-1}$:

\begin{center}
$A^{-1}Ax = A^{-1}b$
\end{center}

$(A^{-1}A)x = A^{-1}b$

$Ix = A^{-1}b$

$x = A^{-1}b$

Hence $A^{-1}b$ is the solution to Ax = b. However, given A has n pivots (is invertible), this solution is unique.

\end{document}