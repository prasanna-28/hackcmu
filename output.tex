\documentclass[11pt,a4paper]{article}
\usepackage[margin=1in]{geometry}
\usepackage{amsmath}

\begin{document}

\section{Inverse Matrices}

An $n\times n$ matrix $A$ is invertible if there is a matrix $A^{-1}$ such that $AA^{-1} = A^{-1}A = I$.

\subsection{Examples}

\begin{itemize}
    \item $A = \begin{bmatrix} 2 & 3 \\ 1 & 4 \end{bmatrix}$, $B = \begin{bmatrix} 4 & -3 \\ -1 & 2 \end{bmatrix}$ are inverses
    \item Since $AB = \begin{bmatrix} 1 & 0 \\ 0 & 1 \end{bmatrix}$, we can write $B = A^{-1}$
    \item Not all $n\times n$ matrices have inverses
    \item $C = \begin{bmatrix} 1 & 2 \\ 2 & 4 \end{bmatrix}$ has no inverse
    \item Since $AB = \begin{bmatrix} 1 & 0 \\ 0 & 1 \end{bmatrix} = I$, $B = A^{-1}$
\end{itemize}

\subsection{Connection to Linear Transformations}

If $T$ is a linear transformation with associated matrix $A$ (in standard bases), then the inverse transformation exists if and only if $A^{-1}$ exists. The matrix of the inverse transformation is $A^{-1}$.

\section{2x2 Inverse Formula}

Let $A = \begin{bmatrix} a & b \\ c & d \end{bmatrix}$. Then if $ad-bc \neq 0$,

\[A^{-1} = \frac{1}{ad-bc} \begin{bmatrix} d & -b \\ -c & a \end{bmatrix}\]

This is a special case of a more general formula (involving the determinant and adjugate matrix) that is more complicated to compute (but we'll see later).

\subsection{Using Cross-Section to Find the Inverse}

Suppose $A$ has inverse $A^{-1}$
Then $AA^{-1} = [A_1 A_2] \begin{bmatrix} x_1 & x_2 \\ y_1 & y_2 \end{bmatrix} = I$

So $Ax_1 = \begin{bmatrix} 1 \\ 0 \end{bmatrix}$ and $Ax_2 = \begin{bmatrix} 0 \\ 1 \end{bmatrix}$

To find $A^{-1}$, we must solve the system $Ax = e_i$ for each standard basis vector $e_i$.

Instead of solving $[A|e_1], [A|e_2], \ldots, [A|e_n]$ individually, we can combine these into one augmented matrix $[A|I]$.

If $A$ doesn't have $n$ pivots (one in each row), then $A^{-1}$ does not exist.
If $A$ has $n$ pivots, then the algorithm produces $[I|A^{-1}]$.

\subsection{Summary}

\begin{itemize}
    \item If $A$ doesn't have $n$ pivots, then $A^{-1}$ doesn't exist
    \item If $A$ has $n$ pivots, then the RREF of $[A|I]$ is $[I|A^{-1}]$
\end{itemize}

\section{Example: Finding $A^{-1}$}

Find $A^{-1}$ if it exists:

\[A = \begin{bmatrix} 2 & 1 & 3 \\ 1 & 0 & 1 \\ 3 & 1 & 4 \end{bmatrix}\]

\begin{align*}
    \begin{bmatrix}
    2 & 1 & 3 & 1 & 0 & 0 \\
    1 & 0 & 1 & 0 & 1 & 0 \\
    3 & 1 & 4 & 0 & 0 & 1
    \end{bmatrix}
    &\sim
    \begin{bmatrix}
    1 & 0 & 1 & 0 & 1 & 0 \\
    0 & 1 & 1 & 1 & -2 & 0 \\
    0 & 0 & 1 & -1 & 1 & 1
    \end{bmatrix} \\
    &\sim
    \begin{bmatrix}
    1 & 0 & 0 & 1 & 0 & -1 \\
    0 & 1 & 0 & 2 & -3 & -1 \\
    0 & 0 & 1 & -1 & 1 & 1
    \end{bmatrix}
\end{align*}

Therefore, $A^{-1} = \begin{bmatrix} 1 & 0 & -1 \\ 2 & -3 & -1 \\ -1 & 1 & 1 \end{bmatrix}$ (Check: $AA^{-1} = I$)

\section{Using Inverses to Solve Systems}

If $A$ exists, we can solve $Ax = b$ by multiplying by $A^{-1}$:

\begin{align*}
    A^{-1}Ax &= A^{-1}b \\
    Ix &= A^{-1}b \\
    x &= A^{-1}b
\end{align*}

However, $A^{-1}$ has subtraction (in its entries), thus solution is unique.

\end{document}