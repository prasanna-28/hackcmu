\documentclass[12pt,a4paper]{article}
\usepackage{amsmath}
\usepackage{amssymb}
\usepackage{geometry}

\geometry{
    top=2cm,
    bottom=2cm,
    left=2cm,
    right=2cm
}

\begin{document}

\section{Inverse Matrices}

An \textit{inverse matrix} $A$ is invertible if there is a matrix $A^{-1}$ such that $AA^{-1} = A^{-1}A = I$.

\subsection{Examples}

\begin{itemize}
    \item $A = \begin{bmatrix} 2 & 3 \\ 1 & 2 \end{bmatrix}$, $B = \begin{bmatrix} 2 & -3 \\ -1 & 2 \end{bmatrix}$ are inverses
    \item Since $AB = \begin{bmatrix} 1 & 0 \\ 0 & 1 \end{bmatrix}$, we can write $B = A^{-1}$
    \item Not all matrices have inverses. $A = \begin{bmatrix} 1 & 2 \\ 2 & 4 \end{bmatrix}$ has no inverse.
\end{itemize}

\subsection{Connection to Linear Transformations}

$A^{-1}$ is the inverse transformation with respect to matrix $A$ (or $T$ at $A \vec{x} = \vec{b}$). The inverse of $T$ is a linear transformation whose associated matrix is $A^{-1}$.

\section{2x2 Inverse Formula}

Let $A = \begin{bmatrix} a & b \\ c & d \end{bmatrix}$, then $T \det(A) \neq 0$:

\[A^{-1} = \frac{1}{ad-bc} \begin{bmatrix} d & -b \\ -c & a \end{bmatrix}\]

This is a special case of a more general formula (involving the determinant and adjugate) that is complicated to compute (will see this later) for larger matrices.

\subsection{Using Cross-Section to Find the Inverse}

Suppose $A$ has inverse $A^{-1}$
Then $AA^{-1} = [A_1 A_2] = I$

So $A\vec{x}_1 = \begin{bmatrix} 1 \\ 0 \end{bmatrix}$ and $A\vec{x}_2 = \begin{bmatrix} 0 \\ 1 \end{bmatrix}$

To find $A^{-1}$, we must solve these systems:
\[A\vec{x}_1 = \vec{e}_1, \quad A\vec{x}_2 = \vec{e}_2\]

Instead of solving $[Ax_1] [Ax_2] = [e_1 e_2]$ individually, we can combine them into one matrix equation:
\[A[x_1, x_2, \ldots] = [e_1, e_2, \ldots]\]

If $A$ doesn't have $n$ pivots (one in each row), then $A$ doesn't exist.
If $A$ has $n$ pivots, then the algorithm will produce $A^{-1}$.

\subsection{Summary}

\begin{itemize}
    \item If $A$ doesn't have $n$ pivots, then $A^{-1}$ doesn't exist.
    \item If $A$ has $n$ pivots, then $A^{-1}$ exists and the RREF is $[I|A^{-1}]$.
\end{itemize}

\section{Example: Finding $A^{-1}$}

Let $A = \begin{bmatrix} 2 & 1 & 3 \\ 0 & 1 & 4 \\ 1 & 0 & 2 \end{bmatrix}$. Find $A^{-1}$, if it exists.

\[
\begin{bmatrix}
2 & 1 & 3 & | & 1 & 0 & 0 \\
0 & 1 & 4 & | & 0 & 1 & 0 \\
1 & 0 & 2 & | & 0 & 0 & 1
\end{bmatrix}
\xrightarrow{\text{Row operations}}
\begin{bmatrix}
1 & 0 & 0 & | & 2 & -1 & -1 \\
0 & 1 & 0 & | & -8 & 5 & 4 \\
0 & 0 & 1 & | & 2 & -1 & -1
\end{bmatrix}
\]

Therefore, 
\[A^{-1} = \begin{bmatrix} 2 & -1 & -1 \\ -8 & 5 & 4 \\ 2 & -1 & -1 \end{bmatrix}\]

(Check: Verify $AA^{-1} = I$)

\subsection{Using Inverses to Solve Systems}

If $A$ exists, we can solve $A\vec{x} = \vec{b}$ by multiplying $A^{-1}\vec{b}$:

\begin{align*}
A\vec{x} &= \vec{b} \\
A^{-1}A\vec{x} &= A^{-1}\vec{b} \\
I\vec{x} &= A^{-1}\vec{b} \\
\vec{x} &= A^{-1}\vec{b}
\end{align*}

Note: $A^{-1}$ has solutions if (and only if) $A$ has $n$ pivots (is invertible). When $A^{-1}$ exists, this solution is unique.

\end{document}