\documentclass[12pt,a4paper]{article}
\usepackage{amsmath}
\usepackage{amssymb}
\usepackage{geometry}

\geometry{
    top=2cm,
    bottom=2cm,
    left=2cm,
    right=2cm
}

\begin{document}

\section{Matrix Inverses}

An \textit{inverse} of a matrix $A$ is denoted $A^{-1}$. A matrix $A$ such that $AA^{-1} = I$.

\subsection{Examples}

\begin{itemize}
    \item $A = \begin{pmatrix} 3 & 1 \\ 1 & 2 \end{pmatrix}$, $B = \begin{pmatrix} \frac{2}{5} & -\frac{1}{5} \\ -\frac{1}{5} & \frac{3}{5} \end{pmatrix}$ are inverses
    \item Since $AB = \begin{pmatrix} 1 & 0 \\ 0 & 1 \end{pmatrix}$, we can write $B = A^{-1}$
    \item Not all matrices have inverses. $A = \begin{pmatrix} 0 & 0 \\ 0 & 1 \end{pmatrix}$ has no inverse
\end{itemize}

\subsection{Connection to Linear Transformations}

Let $T$ be a linear transformation with associated matrix $A$ (in FTLA or FTVS). The inverse of $T$ is a linear transformation with associated matrix $A^{-1}$.

\section{2x2 Inverse Formula}

Let $A = \begin{pmatrix} a & b \\ c & d \end{pmatrix}$, then if $ad-bc \neq 0$:

\[A^{-1} = \frac{1}{ad-bc} \begin{pmatrix} d & -b \\ -c & a \end{pmatrix}\]

There is a general $n \times n$ inverse formula (involving the determinant) but it's complicated to compute (we'll see this later). For now, we'll focus on the 2x2 case.

\subsection{Using Cross-Section to Find the Inverse}

Suppose $A$ has inverse $A^{-1}$. Then $AA^{-1} = I$ and $A^{-1}A = I$.

So $A\begin{pmatrix} x_1 \\ x_2 \end{pmatrix} = \begin{pmatrix} 1 \\ 0 \end{pmatrix}$ and $A\begin{pmatrix} y_1 \\ y_2 \end{pmatrix} = \begin{pmatrix} 0 \\ 1 \end{pmatrix}$

Then $A^{-1} = \begin{pmatrix} x_1 & y_1 \\ x_2 & y_2 \end{pmatrix}$

To find $A^{-1}$, we must solve the system $Ax = e_1$ and $Ay = e_2$. Instead of solving $[Ax], [Ay], [Az]$ individually, we can streamline this process by solving one system at once:

\[A[x_1 \; y_1 \; z_1] = [e_1 \; e_2 \; e_3]\]

If $A$ doesn't have $n$ pivots (one in each row), then $A^{-1}$ does not exist.

If $A$ has $n$ pivots, then the algorithm $\frac{1}{n}[A \; I] \rightarrow [I \; A^{-1}]$ works.

\subsection{Example}

Find $A^{-1}$, if it exists:

\[A = \begin{pmatrix} 2 & 1 & 3 \\ 1 & 0 & 1 \\ 0 & 1 & 2 \end{pmatrix}\]

\[
\begin{pmatrix}
2 & 1 & 3 & | & 1 & 0 & 0 \\
1 & 0 & 1 & | & 0 & 1 & 0 \\
0 & 1 & 2 & | & 0 & 0 & 1
\end{pmatrix}
\]

(Reduced row echelon form steps omitted)

\[
\begin{pmatrix}
1 & 0 & 0 & | & 2 & -1 & -1 \\
0 & 1 & 0 & | & -4 & 3 & 1 \\
0 & 0 & 1 & | & 1 & -1 & 0
\end{pmatrix}
\]

Therefore, $A^{-1} = \begin{pmatrix} 2 & -1 & -1 \\ -4 & 3 & 1 \\ 1 & -1 & 0 \end{pmatrix}$ (Check: $AA^{-1} = I$)

\section{Using Inverses to Solve Systems}

If $A$ exists, we can solve $Ax = b$ by multiplying both sides by $A^{-1}$:

\[A^{-1}Ax = A^{-1}b\]
\[Ix = A^{-1}b\]
\[x = A^{-1}b\]

Here $A^{-1}b$ is a solution to $Ax = b$. However, even if $A$ has $n$ pivots (is invertible), this solution is unique.

\end{document}