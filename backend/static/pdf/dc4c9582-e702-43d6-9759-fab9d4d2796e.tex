\documentclass[12pt,a4paper]{article}
\usepackage[margin=1in]{geometry}
\usepackage{amsmath}

\begin{document}

\section{Inverse Matrices}

An $n\times n$ matrix $A$ is invertible if there is a matrix $A^{-1}$ such that $AA^{-1} = A^{-1}A = I$.

\subsection{Examples}

\begin{itemize}
    \item $A = \begin{bmatrix} 2 & 3 \\ 1 & 2 \end{bmatrix}$, $B = \begin{bmatrix} 2 & -3 \\ -1 & 2 \end{bmatrix}$ are inverses
    \item Since $AB = \begin{bmatrix} 1 & 0 \\ 0 & 1 \end{bmatrix}$, we can write $B = A^{-1}$
    \item Not all matrices have inverses. e.g., $A = \begin{bmatrix} 0 & 0 \\ 3 & 4 \end{bmatrix}$ has no inverse
\end{itemize}

\subsection{Connection to Linear Transformations}

Let $T$ be a linear transformation with associated matrix $A$ (in FTLA's V-W). The inverse of $T$ is a linear transformation with associated matrix $A^{-1}$.

\section{2x2 Inverse Formula}

Let $A = \begin{bmatrix} a & b \\ c & d \end{bmatrix}$, then if $ad-bc \neq 0$:

\[A^{-1} = \frac{1}{ad-bc} \begin{bmatrix} d & -b \\ -c & a \end{bmatrix}\]

Note: $\det A = ad-bc$ is called the \textit{determinant}

This is a special 2x2 inverse formula (extending to determinants which we will see later). For now, this formula is complicated to compute (until we learn how).

\subsection{Using Gauss-Jordan to Find the Inverse}

Suppose $A$ has inverse $A^{-1}$
Then $AA^{-1} = [A|I] \sim [I|A^{-1}]$

So to find $A^{-1}$, we must solve the system $Ax = e_i$ for each $i$.

Instead of solving $[A|e_1], [A|e_2], \ldots, [A|e_n]$ individually, we can solve one augmented matrix $[A|I]$

If $A$ doesn't have $n$ pivots (one in each row), then $A^{-1}$ does not exist.

If $A$ has $n$ pivots, then the algorithm will produce $[I|A^{-1}]$

\subsection{Summary: Finding Inverse Using Gauss-Jordan}
\begin{itemize}
    \item If $A$ doesn't have $n$ pivots, then $A^{-1}$ doesn't exist
    \item If $A$ has $n$ pivots, then the algorithm will produce $[I|A^{-1}]$
\end{itemize}

\section{Example: Finding $A^{-1}$}

Find $A^{-1}$, if it exists:

\[A = \begin{bmatrix} 2 & 1 & 3 \\ 1 & 0 & 1 \\ 3 & 1 & 4 \end{bmatrix}\]

Solution:

\[[A|I] = \begin{bmatrix} 
2 & 1 & 3 & | & 1 & 0 & 0 \\
1 & 0 & 1 & | & 0 & 1 & 0 \\
3 & 1 & 4 & | & 0 & 0 & 1
\end{bmatrix}\]

(Perform row operations...)

\[[I|A^{-1}] = \begin{bmatrix}
1 & 0 & 0 & | & -1 & 3 & -1 \\
0 & 1 & 0 & | & -2 & -5 & 3 \\
0 & 0 & 1 & | & 1 & -2 & 0
\end{bmatrix}\]

Therefore, 

\[A^{-1} = \begin{bmatrix} -1 & 3 & -1 \\ -2 & -5 & 3 \\ 1 & -2 & 0 \end{bmatrix}\]

(Check: Verify $AA^{-1} = I$)

\section{Using Inverses to Solve Systems}

If $A$ exists, we can solve $Ax = b$ by multiplying by $A^{-1}$:

\begin{align*}
A^{-1}Ax &= A^{-1}b \\
Ix &= A^{-1}b \\
x &= A^{-1}b
\end{align*}

Note: $Ax = b$ has solution $x = A^{-1}b$. However, even if $A$ has $n$ pivots (so it is invertible), this solution is no more efficient than Gauss-Jordan elimination.

\end{document}