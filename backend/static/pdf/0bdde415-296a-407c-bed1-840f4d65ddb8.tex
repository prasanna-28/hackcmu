\documentclass[12pt,a4paper]{article}
\usepackage[margin=1in]{geometry}
\usepackage{amsmath}

\begin{document}

\section{Inverse Matrices}

An $n \times n$ matrix $A$ is invertible if there is a matrix $A^{-1}$ such that $AA^{-1} = A^{-1}A = I$.

\subsection{Examples}
\begin{itemize}
    \item $A = \begin{bmatrix} 2 & 3 \\ 1 & 2 \end{bmatrix}$, $B = \begin{bmatrix} 2 & -3 \\ -1 & 2 \end{bmatrix}$ are inverses
    \item Since $AB = \begin{bmatrix} 1 & 0 \\ 0 & 1 \end{bmatrix}$, we can write $B = A^{-1}$
    \item Not all $n \times n$ matrices have inverses
    \item $A = \begin{bmatrix} 0 & 0 \\ 0 & 1 \end{bmatrix}$ has no inverse
    \item $AB = \begin{bmatrix} 0 & 0 \\ 0 & 1 \end{bmatrix} \neq I$
\end{itemize}

\subsection{Connection to Linear Transformations}
$A^{-1}$ is the inverse transformation with respect to $A$ (or $T_A^{-1}$ w.r.t. $T_A$). The inverse of $T$ is a linear transformation whose associated matrix is $A^{-1}$.

\section{2x2 Inverse Formula}

Let $A = \begin{bmatrix} a & b \\ c & d \end{bmatrix}$, then if $ad-bc \neq 0$:

\[A^{-1} = \frac{1}{ad-bc} \begin{bmatrix} d & -b \\ -c & a \end{bmatrix}\]

This is a special case of a more general formula (involving the determinant and adjugate) which is more complicated to compute.

\subsection{Using Gauss-Jordan to Find the Inverse}
Suppose $A$ has inverse $A^{-1}$
Then $AA^{-1} = I$
So $A[A^{-1}] = [I]$

To find $A^{-1}$, we must solve the system $Ax = e_i$ for each $i$.

Instead of solving $[A][x_1] = [e_1]$, $[A][x_2] = [e_2]$, ..., $[A][x_n] = [e_n]$ individually, we can streamline the process by augmenting $A$ with $I$:

$[A|I]$

If $A$ doesn't have $n$ pivots (one in each row), then $A^{-1}$ does not exist.
If $A$ has $n$ pivots, then the algorithm produces $[I|A^{-1}]$.

\subsection{Summary}
\begin{itemize}
    \item If $A$ doesn't have $n$ pivots, then $A^{-1}$ doesn't exist
    \item If $A$ has $n$ pivots, then we obtain $A^{-1}$
\end{itemize}

\section{Example}

Find $A^{-1}$, if it exists:

\[A = \begin{bmatrix} 2 & 4 & -2 \\ 4 & 9 & -3 \\ -2 & -3 & 7 \end{bmatrix}\]

\begin{align*}
\begin{bmatrix}
2 & 4 & -2 & | & 1 & 0 & 0 \\
4 & 9 & -3 & | & 0 & 1 & 0 \\
-2 & -3 & 7 & | & 0 & 0 & 1
\end{bmatrix}
&\xrightarrow{R_2 - 2R_1}
\begin{bmatrix}
2 & 4 & -2 & | & 1 & 0 & 0 \\
0 & 1 & 1 & | & -2 & 1 & 0 \\
-2 & -3 & 7 & | & 0 & 0 & 1
\end{bmatrix} \\
&\xrightarrow{R_3 + R_1}
\begin{bmatrix}
2 & 4 & -2 & | & 1 & 0 & 0 \\
0 & 1 & 1 & | & -2 & 1 & 0 \\
0 & 1 & 5 & | & 1 & 0 & 1
\end{bmatrix} \\
&\xrightarrow{R_3 - R_2}
\begin{bmatrix}
2 & 4 & -2 & | & 1 & 0 & 0 \\
0 & 1 & 1 & | & -2 & 1 & 0 \\
0 & 0 & 4 & | & 3 & -1 & 1
\end{bmatrix}
\end{align*}

Therefore:

\[A^{-1} = \begin{bmatrix} \frac{11}{8} & -\frac{1}{4} & \frac{1}{8} \\ -\frac{5}{8} & \frac{3}{4} & -\frac{1}{8} \\ \frac{3}{4} & -\frac{1}{4} & \frac{1}{4} \end{bmatrix} \quad \text{(Check: $AA^{-1} = I$)}\]

\subsection{Using Inverses to Solve Systems}
If $A$ exists, we can solve $Ax = b$ by multiplying both sides by $A^{-1}$:

\begin{align*}
A^{-1}Ax &= A^{-1}b \\
Ix &= A^{-1}b \\
x &= A^{-1}b
\end{align*}

Note: $A^{-1}b$ has solution $x$ if $b$ (and $A$) have no pivots. However, even if $A$ has $n$ pivots (is invertible), this solution is unique.

\end{document}