\documentclass[12pt,a4paper]{article}
\usepackage[margin=1in]{geometry}
\usepackage{amsmath}

\begin{document}

\section{Inverse Matrices}

An non-singular matrix A is invertible if there is a matrix $A^{-1}$ such that $AA^{-1} = A^{-1}A = I$.

\subsection{Examples}

$A = \begin{bmatrix} 2 & 1 \\ 1 & 3 \end{bmatrix}$, $B = \begin{bmatrix} 3 & -1 \\ -1 & 2 \end{bmatrix}$ are inverses

Since $AB = \begin{bmatrix} 1 & 0 \\ 0 & 1 \end{bmatrix}$ we can write $B = A^{-1}$

Not all matrices have inverses:

$C = \begin{bmatrix} 0 & 0 \\ 0 & 1 \end{bmatrix}$ has no inverse.

Since $AB = \begin{bmatrix} 0 & 0 \\ 0 & 1 \end{bmatrix} \neq I$

\subsection{Connection to Linear Transformations}

$A: T \rightarrow V$ is a linear transformation with associated matrix A (as T is A's V basis)

The inverse of T is a linear transformation with associated matrix $A^{-1}$

\section{2x2 Inverse Formula}

Let $A = \begin{bmatrix} a & b \\ c & d \end{bmatrix}$, then if $ad-bc \neq 0$:

\[A^{-1} = \frac{1}{ad-bc} \begin{bmatrix} d & -b \\ -c & a \end{bmatrix}\]

\textit{Proof:} Show $AA^{-1} = I$ and $A^{-1}A = I$

There is a general nxn inverse formula (involving the determinant and adjugate) but it's complicated to compute (we'll see this later). For now, we'll use row reduction to find the inverse (if it exists).

\section{Using Gauss-Jordan to Find the Inverse}

Suppose A has inverse A$^{-1}$

Then $AA^{-1} = I$

So $[A|I] \rightarrow [I|A^{-1}]$

To find A$^{-1}$, we must solve the system Ax = e$_i$ for each i.

Instead of solving $[A|e_1], [A|e_2], \ldots, [A|e_n]$ individually, we can solve $[A|I]$ once to get all solutions at the same time.

If A doesn't have n pivots (one in each row) then A$^{-1}$ does not exist.

If A has n pivots, then the algorithm $\frac{1}{2}[I|A^{-1}] \rightarrow [A^{-1}|I]$ gives A$^{-1}$.

\subsection{Summary}

\begin{itemize}
    \item If A doesn't have n pivots, then A$^{-1}$ doesn't exist
    \item If A has n pivots, then the RREF is [I|A$^{-1}$]
\end{itemize}

\section{Example}

Find A$^{-1}$, if it exists:

\[A = \begin{bmatrix} 2 & 1 & 3 \\ 4 & 1 & 0 \\ -2 & 2 & 1 \end{bmatrix}\]

\[
\begin{bmatrix}
2 & 1 & 3 & | & 1 & 0 & 0 \\
4 & 1 & 0 & | & 0 & 1 & 0 \\
-2 & 2 & 1 & | & 0 & 0 & 1
\end{bmatrix} \rightarrow
\begin{bmatrix}
1 & 0 & 0 & | & -1/6 & 1/6 & 1/6 \\
0 & 1 & 0 & | & 2/3 & -1/3 & -1/3 \\
0 & 0 & 1 & | & -1/6 & -1/6 & 1/6
\end{bmatrix}
\]

\[A^{-1} = \begin{bmatrix} -1/6 & 1/6 & 1/6 \\ 2/3 & -1/3 & -1/3 \\ -1/6 & -1/6 & 1/6 \end{bmatrix}\]

(Check: Verify $AA^{-1} = I$)

\section{Using Inverses to Solve Systems}

If A is invertible, we can solve Ax = b by multiplying both sides by A$^{-1}$:

\[A^{-1}Ax = A^{-1}b\]
\[x = A^{-1}b\]

Hence $x_i = \sum_j (A^{-1})_{ij} b_j$

However, solving A$^{-1}$ is (in practice) more complicated than solving Ax = b directly.

\end{document}